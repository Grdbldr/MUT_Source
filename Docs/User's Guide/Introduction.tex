\label{chapter:Introduction} This document describes a new \mfu\footnote{\url{https://www.gsienv.com/software/modflow-usg/modflow-usg/}}  development environment which has these features:
\begin{itemize}
    \item We refer to it as Modflow User Tools, or \mut\ for short.
    \item \mut\ is designed to work with a modified version of \mfu,  where a new surface water flow package, called \swf, has been added. Like the Connected Linear Network (\cln) package, the \swf\ package represents a new domain type that is fully-coupled to the 3D groundwater flow (\gwf) domain. There can also be cell-to-cell flows between the \swf\ and \cln\ domains.  The \swf\ domain uses the diffusion-wave approach to simulate 2D surface-water flow. We will refer to this new version of \mfu\ as \mfus\ in this manual.
    \item We currently develop and run it on a \windows\ 10-based computing platform, writing software using the \ifort\ compiler running inside the \vstudio\ interactive development environment, which includes software version control tools through \github.
    \item A text-based approach is used for the \mut\ interface, in which we first develop an input file of instructions that define our \mfus\ project,  then run \mut\ to read it and write a complete \mfus\ data set. \mut\ also writes output files for \tecplot, a third-party visualization software package, which provides a 3D graphical visualization tool to review the model numerical mesh and material properties in the data set. In future, \mut\ could be extended to support other third-party visualization packages, for example the open source program Paraview.
    \item \mut\ can post-process a \mfus\ simulation to provide a \tecplot\ visualization of temporal model results, including hydraulic heads, saturations, water depths and flow budget data.  \textit{If applied to output files which were produced by an earlier version of Modflow, results may be mixed.  It is not our intent here to support all existing Modflow packages, many of which have been superceded.}
\end{itemize}

This document is subivided into these sections:
\begin{description}
    \item[Chapter~\ref{chapter:Installation}]\textbf{Installation and Setup:} How to install \mut, \mfus\ and \tecplot\ and define \windows\ environment variables.
     \item[Chapter~\ref{chapter:ModelBuild}]\textbf{\mut\ Execution and Pre-processing} How to build a \mut\ input file, produce a \mfus\ compatible data set and \tecplot\ compatible output files with \mut, then review the results of the model build with \tecplot.
    \item[Chapter~\ref{chapter:ModelExecution}]\textbf{\mfus\ Execution and Post-Processing} How to run \mfus, convert the output to \tecplot-compatible files with \mut, then visualize them with \tecplot.
    \item[Chapter~\ref{chapter:ModelVerification}]\textbf{Model Verification} Examples used to verify the accuracy of \mfus\ models built using \mut.
    \item[Chapter~\ref{chapter:IllustrativeExample}]\textbf{Illustrative Example} An example which illustrates the use of \mut\ and \mfus\ to simulate variably-saturated, fully-coupled   \gwf-\swf\ flow in a small watershed.
    \item[Appendix~\ref{Appendix:ExcelUseage}]\textbf{\excel\ Database Files} Details about using the provided \excel\ database files, which are currently used to store \mfus\ model material property and solver parameter data sets.
%    \item[Appendix~\ref{Appendix:TecplotUseage}]\textbf{\tecplot\ Useage} Details about using \tecplot\ to visualize output files generated by \mut\ during model build and execution.
        \end{description} 