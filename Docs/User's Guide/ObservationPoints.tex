\label{section:ObservationPoints}
Observation points are used to output cell properties such as simulated hydraulic head or surface water depth during a numerical simulation.  The following instructions can be used to define new observation points:

\ins{observation point}
{
    \squish
    \begin{enumerate}
        \item \str{Name} Observation point name \str{Name}.
        \item \rnum{x} [$L$],\rnum{y} [$L$], \rnum{z} [$L$].  Observation point location as $xyz$ triplet.
    \end{enumerate}
    \squish
}

The example below\footnote{See example \texttt{6\_Abdul\_Prism\_Cell}} shows observation points being defined for both the \gwf\ and \swf\ domains:

        \includegraphics[width=\textwidth]{MUT_ObsPoints.png}

These coordinates are mesh cell locations that were determined using the \tecplot\ Probe Tool.

The following instruction reads observation point data from a {\tt csv} file:

\ins{observation points from csv file}
    {
        \squish
        \begin{enumerate}
        \item \str{FName}  Observtion points {\tt csv} file name.
        \end{enumerate}
          \mut\ uses the file \str{FName} to read the names and $xyz$ coordinates of the observation points.
    }

 The contents of  a {\tt csv} file containing observation point data that was created in \qgis\ are shown below:

        \includegraphics[width=\textwidth]{MUT_ObservationPtsCSV.png}

Note the following:
\begin{itemize}
    \item The first line of the file, which contains the list of field names generated in \qgis, is read and ignored.
    \item The remainder of the file contains observation point data records (id's, names and $xyz$ coordinates) which are added until the end of file is reached.
\end{itemize}

