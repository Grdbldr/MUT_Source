\label{texfile:Units}
By default, \mut\ uses meters and seconds as the units of length and time respectively.  

The instruction \textsf{units of length} is used to change the default value:

\ins{units of length}
    {
        \squish
        \begin{enumerate}
        \item \str{LengthUnit}  The desired unit of length: feet, meters or centimeters.
        \end{enumerate}
        
        So, for example, to use units of centimeters instead of meters, we would put the string 'centimeters' in the input file, which would then be assigned to the string variable \str{LengthUnit}.
    }

When naming input variables in the command description, the following conventions will be used:
\begin{itemize}
    \item Alphanumeric string variable names will begin with
    the string '\str{}'
    \item Real number (i.e. containing a decimal point) variable names will begin with
    the string '\rnum{}'
    \item Integer number (i.e. no decimal point) variable names will begin with
    the string '\inum{}'
\end{itemize}

The instruction \textsf{units of time} is used to change the default value:

\ins{units of time}
    {
        \squish
        \begin{enumerate}
        \item \str{TimeUnit}  The desired unit of time: seconds, minutes, hours, days or years.
        \end{enumerate}

        So, for example, to use units of days instead of seconds, we would put the string 'days' in the input file, which would then be assigned to the string variable \str{TimeUnit}.
    }



