\label{section:Units}
The input files written by \mut\ define the units of length and time that are applied during the \mfus\ model simulation.

By default, \mut\ assigns meters as the units of length and seconds as the units of time. If desired, these instructions can be used to define a different unit system:

\ins{units of length}
    {
        \squish
        \begin{enumerate}
        \item \str{units}  The desired units of length.  
        \end{enumerate}
        \mfus\ currently supports units of feet, meters or centimeters.
        }

\ins{units of time}
    {
        \squish
        \begin{enumerate}
        \item \str{units}  The desired unit of time. \mfus\ currently supports units of seconds, minutes, hours, days or years.
        \end{enumerate}

        So, for example, to use units of days instead of seconds, we would put the string \textsf{days} in the input file,
          which would then be assigned to the string variable \str{units}.
        \squish
    }
    
So, for example, to use units of centimeters and days instead of meters and seconds, we could use the following instructions:
 \begin{verbatim}
    units of length
    centimeters

    units of time
    days
 \end{verbatim}

 \begin{verbatim}
    units of time
    days
 \end{verbatim}
 
\mut\ converts the string variable \str{units} to its numeric equivalent and passes that to \mfus\ through the variables \textsf{ITMUNI} and \textsf{LENUNI}.
