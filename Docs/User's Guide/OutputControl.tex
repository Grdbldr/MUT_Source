\label{section:OutputControl}
This instruction can be used to generate a \mfus\ output control file:

\ins{generate output control file}
    {
    \squish
    \begin{enumerate}
    \item \rnum{t(1)} [$T$].  First output time.
    \item \textbf{...} \\
     \hspace*{-.27in}\rule{0.in}{.24in}  n. \rnum{t(n)} [$T$].  nth output time.
    \end{enumerate}

    An end instruction is required to stop the subtask e.g.:

    {\Large \sf end generate output control file}
    }

The  example \texttt{1\_Abdul\_prism\_cell} generates an output control file with 10 output times using these instructions:
\begin{verbatim}
    ! -----------------------------------Output Control
    generate output control file
        1e-4
        60.
        300.0
        600.0
        900.0
        1200.0
        1500.0
        3000.0
        4500.0
        6000.0
    end generate output control file
\end{verbatim}

The output control file looks like this:
\begin{verbatim}
    # MODFLOW-USG OC file written by Modflow-User-Tools version 1.28
    ATSA NPTIMES   10
      9.999999747378752E-005   60.0000000000000        300.000000000000
       600.000000000000        900.000000000000        1200.00000000000
       1500.00000000000        3000.00000000000        4500.00000000000
       6000.00000000000
    HEAD SAVE UNIT   114
    HEAD PRINT FORMAT 0
    DRAWDOWN SAVE UNIT   115
    DRAWDOWN PRINT FORMAT 0
    PERIOD     1
        DELTAT   1.0000E-03
        TMINAT   1.0000E-05
        TMAXAT    60.00
        TADJAT    1.100
        TCUTAT    2.000
            SAVE HEAD
            PRINT HEAD
            SAVE DRAWDOWN
            SAVE BUDGET
            PRINT BUDGET
    PERIOD     2
        DELTAT   1.0000E-03
        TMINAT   1.0000E-05
        TMAXAT    60.00
        TADJAT    1.100
        TCUTAT    2.000
            SAVE HEAD
            PRINT HEAD
            SAVE DRAWDOWN
            SAVE BUDGET
            PRINT BUDGET
\end{verbatim}

Some key features of this example are:
\begin{itemize}
    \item \mut\ automatically inserts  the  adaptive time-stepping option (\texttt{ATSA}) in the file, defines the number of print times in the simulation (NPTIMES 10) and the list of print (i.e.\ output) times.
    \item Two stress periods were defined and the listed parameters are using the default values.
\end{itemize}

