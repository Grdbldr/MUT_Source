\label{texfile:CLN}
A \cln\ domain is a quasi-3D network of modflow cells which are each defined by individual line segments. The flows between \cln\ cells are governed by either open- or closed-channel flow equations, depending on surrounding conditions, which ultimately provide a pressure head or water depth at each cell.

\mut\ adds the \cln\ process equations to the global
conductance matrix in order to represent fluxes between cells
within the process as well as with cells of other processes.

The current version of \mut\ has the following limitations for the definition of the \cln\ domain:
\begin{enumerate}
    \item General \cln\ domains, where the cell geometry is independent of the \gwf\ mesh and  cell-to-cell connection can be one-to-many or many-to-one are not yet implemented. \mut\ assumes a 1-to-1 connection exists between \cln\ cells and \gwf\ layer 1 cells (i.e. top layer).
    \item \cln\ flows to the \gwf\ domain are implemented, but not  \cln\ flows to the \swf\ domain.
\end{enumerate}

This is sufficient for the short-term purpose of solving the example \texttt{3\_1\_CLN\_for\_SWF}, which compares the use of a \cln\ versus an \swf\ domain for simulating flow in a surface water domain coupled to a \gwf\ domain, but not for solving more general problems of interest.

\subsection{Generating a \cln\ Domain}
A \cln\ domain can be added to the \mfus\ model using this instruction:

\ins{generate cln domain}
    {This subtask is currently limited to defining the \cln\ domain from a single pair of $xyz$ coordinates and a specified number of cells.

    An end instruction is required to stop the subtask e.g.:
    
    {\Large \sf end generate cln domain}
    }

This instruction can be used to define a simple \cln\ domain:

\ins{cln from xyz pair}
    {
        \squish
        \begin{enumerate}
        \item \rnum{x1} [$L$], \rnum{y1} [$L$], \rnum{z1} [$L$]  First $xyz$ coordinate triplet.
        \item \rnum{x2} [$L$], \rnum{y2} [$L$], \rnum{z2} [$L$]  Second $xyz$ coordinate triplet.
        \item \inum{nCells}  Number of cells in the \cln.
        \end{enumerate}
        The 2 given $xyz$ coordinates define the endpoints of a line defining the \cln.  The \cln\ is subdivided into \inum{nCells} individual cells.
    }

In the example \texttt{3\_1\_CLN\_for\_SWF}, the inputs are defined so the \cln\ cells coincide exactly with the top layer of \gwf\ cells as shown below:
\begin{verbatim}
    generate uniform rectangles
    101.0, 101, -0.5   !  Mesh length in X, nRectElements in X, X Offset
    1.0, 1, -0.5   !  Mesh length in Y, nRectElements in Y, Y Offset


    generate layered gwf domain

        top elevation
            elevation from xz pairs
                 -100.0,  3.0
                  200.0,  0.0
            end elevation from xz pairs
        end top elevation

        new layer
            layer name
            Top layer

            uniform sublayering
            1

            elevation from xz pairs
                 -100.0,  2.0
                  200.0, -1.0
            end elevation from xz pairs
        end new layer

    end generate layered gwf domain

    generate cln domain
        cln from xyz pair
              -.5    0.0     2.005
            100.5    0.0     0.995
            101  ! number of new CLN cells

    end generate cln domain

\end{verbatim}

Some key features of this example are:
\begin{itemize}
    \item A template mesh of length 101.0 and with 101 elements is used to define the \gwf\ domain.
    \item The mesh is offset in $x$ by -0.5 m so that \gwf\ and \cln\ cell locations start at $x=0.0$ and end at $x=100$ m.
    \item The top of the \gwf\ domain slopes from $z=3.0$ at $x=-100.0$ to $z=0.0$ at $x=200.0$.  These are extended beyond the limits of the mesh so that the correct elevations are generated at the limits of the domain. Recall that the $x$-range must cover the entire template mesh.
    \item Because \cln\ domains are not necessarily dependent on \gwf\ meshes, it does not use the template mesh, but instead generates a \cln\ domain using the  \texttt{cln from xyz pair} instruction.  The instruction is set up to generate 101 \cln\ cells that slope from $z=2.005$ at $x=-.5$ to $z=0.995$ at $x=100.5$
\end{itemize}

\subsubsection{Cell Connection Properties}  \index{\cln\ Domain ! cell connection properties}

\cln\ domain cells are currently limited to have a direct one-to-one correspondence with cells in the top layer of the \gwf\ domain, as is the case for the simple example \texttt{3\_1\_CLN\_for\_SWF}.  A more general approach, in which \cln\ cells do not have to conform to the \gwf\ or \swf\ meshes will be developed in a subsequent version of \mut.

\cln-\cln\ cells are connected through neighbouring nodes and connection properties are defined by the channel geometry inputs.

\cln-\gwf\ cell vertical connections are defined by element areas.

\subsubsection{Material Properties}  \index{\cln\ Domain ! material properties}
\cln\ domain material properties vary on a zone-by-zone basis.  In the current version of \mut\, the assignment of \cln\ material properties is very rudimentary.

Prior to assigning properties, we need to activate the \cln\ domain using these instructions:
\begin{verbatim}
    active domain
    cln
\end{verbatim}

A lookup table of \cln\ material properties  is provided in the file \texttt{CLN.csv}, located in the \bin\ directory as outlined on page~\pageref{page:userbin}.

In order for \mut\ to access the lookup table, you first need to provide a link to this file using the instruction:

\ins{cln materials database}
    {
        \squish
        \begin{enumerate}
        \item \str{FName}  \cln\ material properties lookup table file name.
        \end{enumerate}
          \mut\ uses the file \str{FName} to look up \cln\ material properties.
    }

Zone selections must first be made using the instructions described on page~\pageref{page:zoneSelect} for the \gwf\ domain.

You can now assign a full set of \cln\ material properties to the current zone selection, as described on page~\pageref{page:zoneSelect}, using this instruction:

\ins{chosen zones use cln material number}
    {
        \squish
        \begin{enumerate}
        \item \inum{MaterialID}  \cln\ material ID number.
        \end{enumerate}
        The unique set of \cln\ material properties with ID number \inum{MaterialID} is retrieved from the lookup table and assigned to the chosen cells.    }

The assigned \cln\ material properties are written to the screen and \texttt{.eco} file:
\begin{verbatim}
    chosen zones use cln material number
    	Assigning all chosen CLN zones properties of material 1, 2D Hillslope 100 m length
    	Geometry:           Rectangular
    	Rectangular Width:       1.0000         METERS
    	Rectangular Height:      1.0000         METERS
    	Direction:          Horizontal
    	Flow Treatment:     Unconfined/Mannings
    	Longitudinal K:         5.48000E-02     METERS   SECONDS^(-1)
\end{verbatim}

You can find detailed information about how to use \dbase\ to modify or define your own lookup tables in Tutorial~\ref{Appendix:ExcelUseage}.

\subsubsection{Initial Conditions}  \index{\cln\ Domain ! initial condition ! initial (starting) depth}
An initial (or starting) head should be assigned to each cell in the \cln\ domain.  This could be an initial guess at the beginning of a transient stress period or a set of hydraulic heads from a previous run.

To assign an initial head to the \cln\ model domain, you must first make a cell selection as described on page~\pageref{page:cellSelect}, then this instruction can be used to calculate an initial (or starting) head for the flow solution given an initial water depth:

\ins{cln initial depth}
    {
        \squish
        \begin{enumerate}
        \item \rnum{Depth} [$L$]  Initial depth.
        \end{enumerate}
          An initial depth of \rnum{Depth} is used to calculate an initial head at each of the chosen cells.
    }

\subsubsection{Boundary Conditions}  \index{\cln\ Domain ! boundary conditions}
To assign boundary conditions  to the \cln\ model domain, you must first make a cell selection as described on page~\pageref{page:cellSelect}.

A constant head boundary condition fixes the head at a \cln\ cell at a given value, allowing water to flow into or out of the \cln\ model domain depending on surrounding conditions.    To assign a uniform constant head to the \cln\ model domain use this instruction:

\ins{cln constant head}
    {
        \squish
        \begin{enumerate}
        \item \rnum{CHead} [$L$]  Constant hydraulic head.
        \end{enumerate}
          An constant hydraulic head  of \rnum{CHead} is assigned to the chosen cells.
    }

A well boundary condition forces  water to flow in or out of the \cln\ model domain at a specified rate.   To add a well  to the \cln\ model domain use this instruction:

\ins{cln well}
    {
        \squish
        \begin{enumerate}
            \item \rnum{PumpRate} [$L^{3}$ $T^{-1}$]  Pumping rate.
        \end{enumerate}
        A pumping  rate of \rnum{PumpRate} is assigned to the chosen cells. Positive pumping rates add water to the domain, negative pumping rates remove water from the domain.
    }


