\label{texfile:CLN} 
A \cln\ domain is a quasi-3D network of modflow cells which are each defined by individual line segments. The flows between \cln\ cells are governed by either open- or closed-channel flow equations, depending on surrounding conditions, which ultimately provide a pressure head or water depth at each cell.

\mut\ adds the \cln\ process equations to the global
conductance matrix in order to represent fluxes between cells
within the process as well as with cells of other processes.

The current version of \mut\ has the following limitations for the definition of the \cln\ domain:
\begin{enumerate}
    \item General \cln\ domains, where the cell geometry is independent of the \gwf\ mesh and  cell-to-cell connection can be one-to-many or many-to-one are not yet implemented. \mut\ assumes a 1-to-1 connection exists between \cln\ cells and \gwf\ layer 1 cells (i.e. top layer).
    \item \cln\ flows to the \swf\ domain are not yet implemented, just \cln\ flows to the \gwf\ domain.
    \item No \cln\ domain boundary conditions have been implemented.  
\end{enumerate}

This is sufficient for the short-term purpose of solving the verification example \texttt{MUT\_Examples$\backslash$3\_1\_CLN\_for\_SWF}, which compares the use of a \cln\ versus an \swf\ domain for simulating flow in a surface water domain coupled to a \gwf\ domain, but not for solving more general problems of interest.
            
\subsection{Generating a \cln\ Domain}
A \cln\ domain can be added to the \mfus\ model using this instruction:

\ins{generate cln domain}
    {This subtask is currently limited to defining the \cln\ domain from a single pair of $xyz$ coordinates and a specified number of cells.

    It reads instructions until an \textsf{end} instruction is found e.g.\:

    {\Large \sf end generate cln domain}
    }

This instruction can be used to define a simple \cln\ domain:

\ins{cln from xyz pair}
    {
        \squish
        \begin{enumerate}
        \item \rnum{x1}, \rnum{y1}, \rnum{z1}  First $xyz$ coordinate triplet.
        \item \rnum{x2}, \rnum{y2}, \rnum{z2}  Second $xyz$ coordinate triplet.
        \item \inum{nCells}  Number of cells in the \cln.
        \end{enumerate}
        The 2 given $xyz$ coordinates define the endpoints of a line defining the \cln.  The \cln\ is subdivided into \inum{nCells} individual cells.
    }

In the verification example \texttt{MUT\_Examples$\backslash$3\_1\_CLN\_for\_SWF}, the inputs are defined so the \cln\ cells coincide exactly with the top layer of \gwf\ cells as shown below:
\begin{verbatim}
    generate uniform rectangles
    101.0, 101   !  Mesh length in X-direction and number of rectangular elements
    1.0, 1   !  Mesh length in Y-direction and number of rectangular elements


    generate layered gwf domain

        top elevation
            elevation from xz pairs
                   0.0, 2.0
                 101.0, 1.0
            end elevation from xz pairs
        end top elevation
        
        ...
 
     generate cln domain
        cln from xyz pair
              0.0    0.5     2.0
            101.0    0.5     1.0
            101  ! number of new CLN cells

    end generate cln domain
\end{verbatim}   

Some key features of this example are:
\begin{itemize}
    \item A template mesh of length 101.0 and with 101 elements is used to define the \gwf\ domain.  
    \item The top of the \gwf\ domain slopes from $z=2.0$ at $x=0.0$ to $z=1.0$ at $x=101.0$.
    \item Because \cln\ domains are not necessarily dependent on \gwf\ meshes, it does not use the template mesh, but instead generates a \cln\ domain using the  \texttt{cln from xyz pair} instruction.  The instruction is set up to generate 101 \cln\ cells that also slope from $z=2.0$ at $x=0.0$ to $z=1.0$ at $x=101.0$
\end{itemize}


\subsubsection{Assigning Material Properties}  \index{\cln\ Domain ! material properties}
\cln\ domain material properties vary on a zone-by-zone basis.  In the current version of \mut\, the assignment of \cln\ material properties is very rudimentary.  

Prior to assigning properties, we need to activate the \cln\ domain using these instructions:
\begin{verbatim}
    active domain
    cln
\end{verbatim}

Zone selections must first be made using the instructions described on page~\pageref{page:zoneSelect} for the \gwf\ domain, then this instructions can be used to assign material properties to the current zone selection:

\ins{chosen zones use cln material number}
    {
        \squish
        \begin{enumerate}
        \item \inum{val}  \cln\ material number.
        \end{enumerate}
          A unique set of \cln\ material properties is retrieved from a lookup table, using the given  material number \inum{val}, and assigned to the chosen zones.
    }

A lookup table of \cln\ material properties  is provided in the file \texttt{qryCLNMaterials.txt}, located in the \bin\ directory as outlined on page~\pageref{page:userbin}.

In order for \mut\ to access the lookup table, you first need to provide a link to this file using the instruction:

\ins{cln materials database}
    {
        \squish
        \begin{enumerate}
        \item \str{file}  \cln\ material properties lookup table file name.
        \end{enumerate}
          \mut\ uses the file \str{file} to look up \cln\ material properties.
    }


You can find detailed information about how to use \dbase\ to modify or define your own lookup tables in Tutorial~\ref{tecfile:DbaseUseage}.

\subsubsection{Initial Conditions}  \index{\cln\ Domain ! initial condition ! initial (starting) depth}
An initial (or starting) head should be assigned to each cell in the \cln\ domain.  This could be an initial guess at the beginning of a transient stress period or a set of hydraulic heads from a previous run.

To assign an initial head to the \cln\ model domain, you must first make a cell selection as described on page~\pageref{page:cellSelect}, then this instruction can be used to calculate an initial (or starting) head for the flow solution given an initial water depth:

\ins{cln initial depth}
    {
        \squish
        \begin{enumerate}
        \item \rnum{value}  Initial depth [$L$].
        \end{enumerate}
          An initial depth of \rnum{value} is used to calculate an initial head at each of the chosen cells.
    }

\subsubsection{Boundary Conditions}  \index{\cln\ Domain ! boundary conditions}
No \cln\ domain boundary conditions have been implemented in the current version of \mut. 

In the verification example \texttt{MUT\_Examples$\backslash$3\_1\_CLN\_for\_SWF}, all boundary conditions are applied to the \gwf\ domain.


